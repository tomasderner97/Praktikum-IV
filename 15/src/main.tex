\documentclass{protokol}
\usepackage[T1]{fontenc}
\usepackage{chemformula}

\leftheader{Studium atomových emisních spekter}
\centerheader{Praktikum IV}
\rightheader{Tomáš Derner}

\begin{document}

    \section*{Úkol}

    \begin{enumerate}

        \item S použitím spektra rtuti zkalibrujte hranolový spektrometr.
        Pro vyloučení hrubých chyb vyneste kalibrační křivku ihned do grafu.
        \item Ověřte vlnové délky sodíkových dubletů (alespoň tří).
        \item Na základě pozorování sodíkových dubletů diskutujte rozlišovací schopnost spektrometru.
        Diskutujte přesnost takto určené rozlišovací schopnosti.
        \item Prohlédněte si spektra výbojek s náplní \ch{He}, \ch{Ne}, \ch{Ar}, \ch{N2} a \ch{CO2}.
        Určete vlnové délky nejjasnějších čar.
        Porovnejte s tabulkovými hodnotami.
        \item Změřte vlnové délky čar \ch{H_{$\alpha$}}, \ch{H_{$\beta$}}, \ch{H_{$\gamma$}} Balmerovy serie vodíkového spektra.
        Vypočítejte Rydbergovu konstantu.

    \end{enumerate}

    \section*{Teorie}

    V této úloze studujeme atomová emisní spektra plynů.
    Využíváme k tomu hranolový spektrometr Hilgerova typu, jehož detailní popis je uveden ve studijním textu~\cite{pokyny}.

    Ve viditelném emisním spektru vodíku jsou pozorovatelné čtyři čáry \ch{H_{$\alpha$}} (červená), \ch{H_{$\beta$}} (modrozelená), \ch{H_{$\gamma$}} (modrá) a \ch{H_{$\delta$}} (fialová).
    Tyto čáry jsou součástí tzv. Balmerovy série, pro vlnočty jejíž spektrálních čár platí vztah
    \begin{equation} \label{eq:rydberg}
        \sigma = \frac{1}{\lambda} = R \left( \frac{1}{4} - \frac{1}{n^2} \right),
    \end{equation}
    kde $R$ je Rydbergova konstanta a $n = 3, 4, 5, 6$ jsou přirozená čísla odpovídající jednotlivým čarám.



    \section*{Výsledky}



    \section*{Diskuse}



    \section*{Závěr}



    \begin{thebibliography}{}

        \bibitem{pokyny}
        Pokyny k měření ``Studium atomových spekter'', dostupné z\\ \url{https://physics.mff.cuni.cz/vyuka/zfp/_media/zadani/texty/txt_415.pdf}, 12.\,11.\,2019

    \end{thebibliography}

\end{document}