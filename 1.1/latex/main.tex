\documentclass{scirep}

\leftheader{Detektor ATLAS}
\centerheader{Parktikum IV}
\rightheader{Tomáš Derner}

\begin{document}

    \section*{Úkol}

    \begin{enumerate}

        \item Zpracujte přibližně 50 událostí z detektoru ATLAS programem HYPATIA.
        \item Pomocí programu ROOT zobrazte histogram invariantních hmotností pro různě velké statistické soubory.
        \item Identifikujte výrazné píky a přiřaďte je očekávaným částicím.
        \item Zjistěte chybu střední hodnoty invariantní hmotnosti Z bozonu pro různě velké statistické soubory.
        \item Vyneste zjištěné chyby do grafu jako funkci počtu událostí a srovnejte je s očekávanou závislostí.
        \item Interpretujte výsledky statistického testu pro nové částice a rozhodněte, jestli byl učiněn objev.

    \end{enumerate}

    \section*{Teorie}

    \[ a = \SI{0,1}{A} \]

    \section*{Výsledky}

    \section*{Diskuse}

    \section*{Závěr}

    \begin{thebibliography}{}

        \bibitem{pokyny}
        Pokyny k měření ``Objevování částic v detektoru ATLAS v CERN'', dostupné z\\ \url{https://physics.mff.cuni.cz/vyuka/zfp/_media/zadani/texty/txt_401.pdf}, 4.\,12.\,2019

    \end{thebibliography}

\end{document}