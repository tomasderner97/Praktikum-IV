\documentclass{scirep}

\leftheader{Sternův-Gerlachův experiment}
\centerheader{Praktikum IV}
\rightheader{Tomáš Derner}

\begin{document}

    \section*{Úkol}

    \begin{enumerate}

        \item Zkontrolujte vakuum v aparatuře a při dosažení potřebného tlaku zprovozněte detektor atomů draslíku a pícku.
        Sledujte zbytkový proud detektoru a v případě potřeby vyčistěte povrch emisní elektrody doporučeným postupem.
        \item Umístěte detektor do maxima nevychýleného svazku (poloha $\si{7,70}$ indikátoru otáček, $U_x=\SI{583}{mV}$) a sledujte nárůst intenzity svazku s teplotou (použitý rozsah nanoampérmetru je \SI{0,1}{nA}).
        Při dosažení signálu z detektoru cca $\SI{2}{V}$ ohřev pícky zastavte (pro stabilizaci této teploty je nyní potřebné střídavé napětí kolem $\SI{6}{V}$, čerstvá náplň draslíku dává tento signál při cca \SI{160}{\celsius}).
        Překontrolujte, že je napětí z detektoru alespoň a zvolte vhodný rozsah Kanálu 1 (výchozí je $\SI{\pm 5}{V}$).
        \item Pomocí programu Stern-Gerlach proměřte prostorový profil atomového svazku při nulovém magnetickém poli.
        \item Tímtéž programem proměřte profily při magnetizačních proudech $\SI{250}{mA}$ ($B=\SI{0,175}{T}$), $\SI{350}{mA}$ ($B=\SI{0,265}{T}$), $\SI{500}{mA}$ ($B=\SI{0,383}{T}$), $\SI{700}{mA}$ ($B=\SI{0,530}{T}$) a $\SI{1000}{mA}$ ($B=\SI{0,675}{T}$).
        \item Naměřené hodnoty rozštěpení svazku vyneste do grafu závislosti $\frac{\partial B}{\partial z}$ na veličině $q = 3ue - \frac{C}{ue}$ (podle vztahu (15) a obr. 8 studijního textu) a regresí určete hodnotu Bohrova magnetonu.
        Diskutujte přesnost metody a citlivost na vynechání některých hodnot.

    \end{enumerate}

    \section*{Teorie}

    V tomto praktiku se zabýváme Stern-Gerlachovým experimentem, jedním z definujících experimentů kvantové mechaniky.
    Pro měření využíváme aparaturu od firmy PHYWE Gottingen a vakuovou soustavu od firmy VAKUUM Praha.

    Aparatura sestává z pícky, která slouží jako zdroj atomů draslíku, které jsou v tomto experimentu použité namísto stříbra, magnetického analyzátoru a Langmuirova – Taylorova detektoru.
    Magnetický analyzátor tvoří v cestě produkovaných atomů draslíku velký gradient magnetického pole, který způsobuje stočení dráhy atomů v závislosti na spinovém stavu jejich valenčního elektronu.
    Abychom mohli detekovat polohy dopadajících atomů, je detektor umístěn na posuvné dráze kolmé na směr dráhy atomů.
    Podrobnější popis použité aparatury lze nalézt ve studijním textu~\cite{pokyny}.

    Pro převod měřeného napětí $U_x$ [V] na polohu detektoru $x$ [mm] lze použít lineární závislost $x = \num{21,20}  U_x + \num{1,90}$.


    \section*{Výsledky}

    Po zkontrolování kvality vakua jsme zapnuli přísun proudu do pícky a pozorovali narůstající odezvu detektoru.
    Pozorovaná závislost je uvedena v tabulce~\ref{tab:heating} a grafu na obrázku~\ref{fig:heating}.

    Pomocí přepínání polarity proudu tekoucího magnetickým analyzátorem za postupného snižování jeho hodnoty jsme demagnetizovali jádro analyzátoru.


    \VAR{tab_heating}
    \VAR{fig_heating}

    \section*{Diskuse}

    \section*{Závěr}

    \begin{thebibliography}{}

        \bibitem{pokyny}
        Pokyny k měření ``Prostorové kvantování magnetického momentu atomu (Sternův-Gerlachův experiment)'', dostupné z\\ \url{https://physics.mff.cuni.cz/vyuka/zfp/_media/zadani/texty/txt_411.pdf}, 27.\,11.\,2019

    \end{thebibliography}

\end{document}